% Created 2019-12-21 Sat 14:59
% Intended LaTeX compiler: pdflatex
\documentclass[11pt]{article}
\usepackage[utf8]{inputenc}
\usepackage[T1]{fontenc}
\usepackage{graphicx}
\usepackage{grffile}
\usepackage{longtable}
\usepackage{wrapfig}
\usepackage{rotating}
\usepackage[normalem]{ulem}
\usepackage{amsmath}
\usepackage{textcomp}
\usepackage{amssymb}
\usepackage{capt-of}
\usepackage{hyperref}
\author{Kamil Vojanec}
\date{\today}
\title{Modelování a simulace}
\hypersetup{
 pdfauthor={Kamil Vojanec},
 pdftitle={Modelování a simulace},
 pdfkeywords={},
 pdfsubject={},
 pdfcreator={Emacs 26.3 (Org mode 9.3)}, 
 pdflang={English}}
\begin{document}

\maketitle
\tableofcontents

\section{Úvod}
\label{sec:org7c899c6}
\subsection{Základní pojmy}
\label{sec:org6010b3b}
\subsubsection{Systém}
\label{sec:org04e1638}
Je soubor elementárních částí, které mají mezi sebou určíté vazby.
\subsubsection{Model}
\label{sec:org0f9109f}
Napodobenina systému jiným systémem
\subsubsection{Modelování}
\label{sec:org5b4882e}
Vyrváření modelů systémů
\subsubsection{Simulace}
\label{sec:org776599c}
Získávání nových znalostí o systémz experimentováním s jeho modelech

\subsection{Základní etapy modelování a simulace}
\label{sec:org8120356}
\begin{enumerate}
\item Vytvoření abstraktního modelu
\item Vytvoření simulačního modelu
\item Verifikace a validace
\item Simulace
\item Analýza a interpretace výsledků
\end{enumerate}

\subsection{Formální definice systému}
\label{sec:orgd2bc696}
Systém \(S\) je dvojice
\[S = (U, R)\]

kde:
\subsubsection{\(U\) je konečná množina prvků systému:}
\label{sec:orga1e0e41}
\[U = \{u_1, u_2, \dots, u_N\}\]

\subsubsection{Prvek systému \(u = (X, Y)\):}
\label{sec:org913f7b5}
\begin{enumerate}
\item \(X\) je množina všech vstupních proměnných
\label{sec:org2e06c6d}
\item \(Y\) je množina všech výstupních proměnných
\label{sec:org784c23c}
\end{enumerate}

\subsubsection{Charakteristika systému \(R\) je množina všech propojení}
\label{sec:org39f3d77}
\[R = \bigcup^{N}_{i,j=1} R_{ij}\]

\subsubsection{Propojení prvku \(u_i\) s prvkem \(u_j\):}
\label{sec:org23b5178}
\[R_{ij} \subseteq Y_i \times X_j\]

\subsection{Čas}
\label{sec:org8edafa0}
Rozlišujeme tři typy času:
\begin{enumerate}
\item Reálný času
Probíháý skutečný děj v reálném systému
\item Modelový čas
Časová osa modelu, nemusí být synchronní s reálným časem
\item Strojový čas
Čas spotřebovaný na výpočet programu
\end{enumerate}

\subsection{Časová množina}
\label{sec:org97e4b41}
Množina všech časových okamžiků, ve kterých jsou definovány hodnoty vstupních,
stavových a výstupních proměnných prvku systému

Může být diskrétní nebo spojitá\footnote{Na číslicovém počítači se vždy diskretizuje}

\subsection{Chování systému}
\label{sec:org7d926d1}
Každému časovému průběhu vstupu přiřazuje časový průběh výstupu. Lze definovat
jako zobrazení.

Systémy považujeme za systémy se stejným chováním, vyvolají-li stejné podněty u
obou systémů stejné reakce.

Izomorfní systémy jsou systémy, kde prvky systémů navzájem můžeme přiřadit 1:1.

Homomorfní systémy jsou systémy, kde prvky systémů navzájem můžeme přiřadit N:1,
ale ne naopak.

\subsection{Okolí systému}
\label{sec:orgabb5218}
Okolí systému zahrnuje vše, co má vliv na chování systému, ale není jeho
součástí.

Systémy můžeme rozdělit na uzavřené (nekomunikují s okolím nebo jej zanedbábají)
a otevřene (mají definované vstupy a výstupy).

\subsection{Klasifikace prvků systémů}
\label{sec:org926e6a0}
\subsubsection{Klasifikace 1}
\label{sec:orgbffac87}
Prvky se spojitým chováním vs prvky s diskrétním chováním
\subsubsection{Klasifikace 2}
\label{sec:org991a254}
Prvky s deterministickým chováním vs prvky s nedeterministickým chováním

\subsection{Klasifikace systémů}
\label{sec:orgac56ac7}
Typ systému závisí na typu jeho prvků
\subsubsection{Klasifikace 1}
\label{sec:org50f6995}
Spojité, Diskrétní, Kombinované
\subsubsection{Klasifikace 2}
\label{sec:orge6d7110}
Deterministické, nedeterministické

\subsection{Simulace}
\label{sec:org89a95a7}
Cílem simulace je získat nové informace o chování systému v závislosti na
vstupních veličinách a na hodnotách parametrů.

Opakovaně vyhodnocujeme model, tak dlouho, dokud nezískáme dostatek informací o
chování systému nebo dokud nenalezneme takové hodnoty parametrů, pro něž má
systém žádané chování.

\subsection{Typy simulace podle modelu}
\label{sec:orgc012bc4}
Spojitá, diskrétní, kombinovaná. Kvalitativní, kvantitativní.

\subsection{Typy simulace podle simulátoru}
\label{sec:orgc85885b}
Na počítači, real-time, paralelní a distribuovaná

\subsection{Verifikace modelu}
\label{sec:orgfd59ab0}
Měla by předcházet simulaci. Ověťujeme, zda simulační model odpovídá 1:1
(izomorfně) abstraktnímu modelu.

\subsection{Validace modelu}
\label{sec:org08bfb94}
Jedná se o jeden z nejobtížnějších problémů modelování. Vyžaduje neustálou
konfrontaci informací a modelovaném systému a dat ze simulovaného systému.

Nelze absolutně dokázat přesnost modelu. Pokud chování modelu neodpovídá
předpokládanému chování systému, musíme model modifikovat.

\section{Modely}
\label{sec:orgfa8fcf7}
\subsection{Abstraktní model}
\label{sec:org9d2a4cc}
Nepostihuje reálný svět v celé komplikovanosti. Zajímá se jen o ohraničené,
vhodně zvolené části. Identifikuje vhodné sloky systému.

Pro sestavení abstraktního modelu potřebujeme studovat systém pro určitá
kritéria a závislosti, predikovat chování systému za určirých podmínek,
analyzovat faktory, které jsou pro činnost systému nejvýznamnější a nalézt
takové kombinace parametrů, které vedou k nejlepší odezvě systému.

\subsection{Simulační model}
\label{sec:org9e82f0b}
Jedná se o abstraktní model zapsaný formou programu v nějakém programovacím
jazyce.

\subsection{Konceptuální modely}
\label{sec:org5571070}
Jejich komponenty zatím nebyly přesně popsány. Používají se v počáteční fázi
modelování pro ujasnění souvislostí a komunikaci v týmu.

Má formu textu nebo obrázků.

\subsection{Deklarativní modely}
\label{sec:org075d155}
Popisuje přechody mezi stavy systému.

Je definován stavy a událostmi, které způsobí přechod z jednoho stavu do
druhého.

Je vhodný především pro diskrétní modely.

Například konečné automaty, petriho sítě.

\subsection{Funkcionální modely}
\label{sec:org98377c2}
Grafy zobrazující funkce a proměnné. Uzel grafu je funkce nebo proměnná.

Například systémy hromadné obsluhy, bloková schemata, grafy signálových toků.

\subsection{Modely popsané rovnicemi}
\label{sec:org9b01199}
Používají se algebraické, diferenciální, diferenční rovnice.

Mívají podobu neorientovaných grafů.

Například elektrická schemata, systémy dravec kořist, kyvadla, logistické
systémy, chaos.

\subsection{Prostorové modely}
\label{sec:org7ff923a}
Rozdělují systém na prostorově menší ohraničené podsystémy.

Například parciální diferenciální rovnice difuze nebo proudění, celulární
automaty, mechanické modely těles.

\subsection{Multimodely}
\label{sec:orge31f870}
Složeny z různých typů modelů, obvykle heterogenních.

Například kombinované modely, fuzzy modely, propojené simulační systémy.
\section{Simulační nástroje}
\label{sec:orgfc9b30f}
Mohou používat klasické programovací jazyky samy o sobě, nebo s využitím
knihoven. Existují specializované simulační jazyky (Simula67, Modellica\ldots{})

\subsection{Simulační jazyky}
\label{sec:org785e86d}
Poskytují prostředky usnadňující efektivní popis struktury a chování modelů a
také popis simulačních experimentů.

Výhodou je jednodušší popis modelu a možnost automatické kontroly popisu modelu.

Nevýhodami jsou náklady na vytvoření překladače, údržbu, výuku\ldots{}

Celkově nejsou příliš používány.

\section{Modelování náhodných procesů}
\label{sec:orgb5c09a2}
\subsection{Náhodná proměnná}
\label{sec:orgf9a2294}
Náhodná proměnná je taková veličina, která jako výsledek pokusů může nabýt
nějakou hodnotu, přičemž předem nevíme jakou.

Rozlišujeme diskrétní a spojité

Náhodné veličiny můžeme zadat distribuční funkcí nebo rozdělením
pravděpodobnosti.

\subsection{Diskrétní rozdělení pravděpodobnosti}
\label{sec:orge2c187b}
Určuje vztah mezi možnými hodnotami náhodné veličiny \(x_i\) a jim příslušejícími
pravděpodobnostmi \(p_i = P(X = x_i)\).

Obecně platí vztah:
\begin{equation}
    \sum_{i=1}^\infty p_i = 1
\end{equation}

Lze definovat například tabulkou pravděpodobnosti pro všechny možné hodnoty
náhodné proměnné, suma všech pravděpodobností musí být 1.

\subsection{Diskrétní distribuční funkce}
\label{sec:org45694d5}
Distribuční funkce náhodné veličiny \(X\) je funkce:
\begin{equation}
    F(x) = P(X \leq x)
\end{equation}

Kde \$P(X \(\le\) x) je pravděpodobnost toho, že náhodná veličina X nabude hodnoty
menší nebo rovnu zvolenému x.

Platí vztah:
\begin{equation}
    F(x) = \sum_{x_i \leq x} p_i
\end{equation}

Kraf distribuční funkce pro diskrétní náhodné proměnné je po částech konstantní.

\subsection{Distribuční funkce spojité náhodné proměnné}
\label{sec:orgd1bf09b}
\begin{equation}
    F(x) = P(X \leq x) = \int_{-\infty}^x f(x) dx
\end{equation}

Distribuční fuknce je neklasající. Roste od 0 do \(\infty\)

Pravděpodobnost hodnoty \(x\) pro \(a \leq x\) a \(x \leq b\) je rozdíl hodnot
distribuční funkce v bodech \(b\) a \(a\). Vyjádřeno vzorcem:
\begin{equation}
    P(a \leq X \leq b) = F(b) - F(a)
\end{equation}

\subsection{Hustota pravděpodobnosti spojité náhodné proměnné}
\label{sec:org1da2fcd}
\begin{itemize}
\item Vždy větší než 0
\item Vyjádřena jako derivace distribuční funkce
\item Integrál funkce hustoty pravděpodobnosti je vždy 1
\item Hodnota pravděpodobnosti pro \(x\) mezi \(a\) a \(b\) je dána jako intergrál hustota
pravděpodobnosti mezi \(a\) a \(b\)
\end{itemize}

\subsection{Poissonovo rozložení}
\label{sec:orgd954f00}
Diskrétní rozložení udávající počet nějakých událostí za jednotku času. Vzorec:
\begin{equation}
p_i = \frac{\lambda^i}{i!} e^{-\lambda}, \lambda > 0, i \in {0, 1, 2,\dots}
\end{equation}
Kde \(E(x) = \lambda, D(x) = \lambda\)

\subsection{Rovnoměrné rozložení}
\label{sec:orgf57cd50}
Označujeme \(R(a, b)\)

Hodnota distribuční funkce lineárně roste mezi body \(a\) a \(b\). Hodnota funkce
hustoty pravděpodobnosti je konstatní.

\subsection{Exponenciální rozložení}
\label{sec:orga338bfe}
Používá se pro dobu mezi dvěma událostmí.
\section{Diskrétní simulace}
\label{sec:org791c61b}
\subsection{Úvod}
\label{sec:orgf9d3306}
\subsection{Petriho sítě}
\label{sec:org4538d76}
\subsection{Systémy hromadné obsluhy}
\label{sec:orgd63702c}
\subsection{SIMLIB}
\label{sec:orge43e6a1}

\section{Spojitá simulace}
\label{sec:org5bf3891}
\subsection{Úvod}
\label{sec:org0481c6b}
\subsection{Numerické metody}
\label{sec:orgfcc31ff}
\subsection{SIMLIB}
\label{sec:org8aba027}

\section{Kombinovaná simulace}
\label{sec:org184a42c}
\subsection{Úvod}
\label{sec:org0b27a8f}
\subsection{SIMLIB}
\label{sec:org417509e}
\end{document}
