% Created 2019-12-21 Sat 14:27
% Intended LaTeX compiler: pdflatex
\documentclass[11pt]{article}
\usepackage[utf8]{inputenc}
\usepackage[T1]{fontenc}
\usepackage{graphicx}
\usepackage{grffile}
\usepackage{longtable}
\usepackage{wrapfig}
\usepackage{rotating}
\usepackage[normalem]{ulem}
\usepackage{amsmath}
\usepackage{textcomp}
\usepackage{amssymb}
\usepackage{capt-of}
\usepackage{hyperref}
\author{Kamil Vojanec}
\date{\today}
\title{Modelování a simulace}
\hypersetup{
 pdfauthor={Kamil Vojanec},
 pdftitle={Modelování a simulace},
 pdfkeywords={},
 pdfsubject={},
 pdfcreator={Emacs 26.3 (Org mode 9.3)}, 
 pdflang={English}}
\begin{document}

\maketitle
\tableofcontents

\section{Úvod}
\label{sec:orgd762b9f}
\subsection{Základní pojmy}
\label{sec:orga5ed381}
\subsubsection{Systém}
\label{sec:org2d91471}
Je soubor elementárních částí, které mají mezi sebou určíté vazby.
\subsubsection{Model}
\label{sec:org8a16c2f}
Napodobenina systému jiným systémem
\subsubsection{Modelování}
\label{sec:orgb8768c7}
Vyrváření modelů systémů
\subsubsection{Simulace}
\label{sec:org17d9fc4}
Získávání nových znalostí o systémz experimentováním s jeho modelech

\subsection{Základní etapy modelování a simulace}
\label{sec:org60746fe}
\begin{enumerate}
\item Vytvoření abstraktního modelu
\item Vytvoření simulačního modelu
\item Verifikace a validace
\item Simulace
\item Analýza a interpretace výsledků
\end{enumerate}

\subsection{Formální definice systému}
\label{sec:orgc78cdb8}
Systém \(S\) je dvojice
\[S = (U, R)\]

kde:
\subsubsection{\(U\) je konečná množina prvků systému:}
\label{sec:orgda291ef}
\[U = \{u_1, u_2, \dots, u_N\}\]

\subsubsection{Prvek systému \(u = (X, Y)\):}
\label{sec:orgff82b24}
\begin{enumerate}
\item \(X\) je množina všech vstupních proměnných
\label{sec:orgb126f38}
\item \(Y\) je množina všech výstupních proměnných
\label{sec:orgc59f790}
\end{enumerate}

\subsubsection{Charakteristika systému \(R\) je množina všech propojení}
\label{sec:org6212db6}
\[R = \bigcup^{N}_{i,j=1} R_{ij}\]

\subsubsection{Propojení prvku \(u_i\) s prvkem \(u_j\):}
\label{sec:orgd1da6ac}
\[R_{ij} \subseteq Y_i \times X_j\]

\subsection{Čas}
\label{sec:org2bc4f11}
Rozlišujeme tři typy času:
\begin{enumerate}
\item Reálný času
Probíháý skutečný děj v reálném systému
\item Modelový čas
Časová osa modelu, nemusí být synchronní s reálným časem
\item Strojový čas
Čas spotřebovaný na výpočet programu
\end{enumerate}

\subsection{Časová množina}
\label{sec:orga8fd394}
Množina všech časových okamžiků, ve kterých jsou definovány hodnoty vstupních,
stavových a výstupních proměnných prvku systému

Může být diskrétní nebo spojitá\footnote{Na číslicovém počítači se vždy diskretizuje}

\subsection{Chování systému}
\label{sec:orgc320c3b}
Každému časovému průběhu vstupu přiřazuje časový průběh výstupu. Lze definovat
jako zobrazení.

Systémy považujeme za systémy se stejným chováním, vyvolají-li stejné podněty u
obou systémů stejné reakce.

Izomorfní systémy jsou systémy, kde prvky systémů navzájem můžeme přiřadit 1:1.

Homomorfní systémy jsou systémy, kde prvky systémů navzájem můžeme přiřadit N:1,
ale ne naopak.

\subsection{Okolí systému}
\label{sec:org94b95fa}
Okolí systému zahrnuje vše, co má vliv na chování systému, ale není jeho
součástí.

Systémy můžeme rozdělit na uzavřené (nekomunikují s okolím nebo jej zanedbábají)
a otevřene (mají definované vstupy a výstupy).

\subsection{Klasifikace prvků systémů}
\label{sec:org6bd1aea}
\subsubsection{Klasifikace 1}
\label{sec:orgc9dc767}
Prvky se spojitým chováním vs prvky s diskrétním chováním
\subsubsection{Klasifikace 2}
\label{sec:orgbc00676}
Prvky s deterministickým chováním vs prvky s nedeterministickým chováním

\subsection{Klasifikace systémů}
\label{sec:orgd4ae8a6}
Typ systému závisí na typu jeho prvků
\subsubsection{Klasifikace 1}
\label{sec:org32b469d}
Spojité, Diskrétní, Kombinované
\subsubsection{Klasifikace 2}
\label{sec:org1c2b8c6}
Deterministické, nedeterministické

\subsection{Simulace}
\label{sec:org4e63120}
Cílem simulace je získat nové informace o chování systému v závislosti na
vstupních veličinách a na hodnotách parametrů.

Opakovaně vyhodnocujeme model, tak dlouho, dokud nezískáme dostatek informací o
chování systému nebo dokud nenalezneme takové hodnoty parametrů, pro něž má
systém žádané chování.

\subsection{Typy simulace podle modelu}
\label{sec:org6e1555d}
Spojitá, diskrétní, kombinovaná. Kvalitativní, kvantitativní.

\subsection{Typy simulace podle simulátoru}
\label{sec:orga727fa6}
Na počítači, real-time, paralelní a distribuovaná

\subsection{Verifikace modelu}
\label{sec:orgdcf035a}
Měla by předcházet simulaci. Ověťujeme, zda simulační model odpovídá 1:1
(izomorfně) abstraktnímu modelu.

\subsection{Validace modelu}
\label{sec:org6d51eaf}
Jedná se o jeden z nejobtížnějších problémů modelování. Vyžaduje neustálou
konfrontaci informací a modelovaném systému a dat ze simulovaného systému.

Nelze absolutně dokázat přesnost modelu. Pokud chování modelu neodpovídá
předpokládanému chování systému, musíme model modifikovat.

\section{Modely}
\label{sec:orgdd4e12b}
\subsection{Abstraktní model}
\label{sec:org1a8d714}
Nepostihuje reálný svět v celé komplikovanosti. Zajímá se jen o ohraničené,
vhodně zvolené části. Identifikuje vhodné sloky systému.

Pro sestavení abstraktního modelu potřebujeme studovat systém pro určitá
kritéria a závislosti, predikovat chování systému za určirých podmínek,
analyzovat faktory, které jsou pro činnost systému nejvýznamnější a nalézt
takové kombinace parametrů, které vedou k nejlepší odezvě systému.

\subsection{Simulační model}
\label{sec:orgcd26dbe}
Jedná se o abstraktní model zapsaný formou programu v nějakém programovacím
jazyce.

\subsection{Konceptuální modely}
\label{sec:orgfb58cfc}
Jejich komponenty zatím nebyly přesně popsány. Používají se v počáteční fázi
modelování pro ujasnění souvislostí a komunikaci v týmu.

Má formu textu nebo obrázků.

\subsection{Deklarativní modely}
\label{sec:org0dbf712}
Popisuje přechody mezi stavy systému.

Je definován stavy a událostmi, které způsobí přechod z jednoho stavu do
druhého.

Je vhodný především pro diskrétní modely.

Například konečné automaty, petriho sítě.

\subsection{Funkcionální modely}
\label{sec:org6fedb80}
Grafy zobrazující funkce a proměnné. Uzel grafu je funkce nebo proměnná.

Například systémy hromadné obsluhy, bloková schemata, grafy signálových toků.

\subsection{Modely popsané rovnicemi}
\label{sec:org5639dd6}
Používají se algebraické, diferenciální, diferenční rovnice.

Mívají podobu neorientovaných grafů.

Například elektrická schemata, systémy dravec kořist, kyvadla, logistické
systémy, chaos.

\subsection{Prostorové modely}
\label{sec:orge30362d}
Rozdělují systém na prostorově menší ohraničené podsystémy.

Například parciální diferenciální rovnice difuze nebo proudění, celulární
automaty, mechanické modely těles.

\subsection{Multimodely}
\label{sec:org07bd3a8}
Složeny z různých typů modelů, obvykle heterogenních.

Například kombinované modely, fuzzy modely, propojené simulační systémy.
\section{Simulační nástroje}
\label{sec:org3b4330c}
Mohou používat klasické programovací jazyky samy o sobě, nebo s využitím
knihoven. Existují specializované simulační jazyky (Simula67, Modellica\ldots{})

\subsection{Simulační jazyky}
\label{sec:org7e78be1}
Poskytují prostředky usnadňující efektivní popis struktury a chování modelů a
také popis simulačních experimentů.

Výhodou je jednodušší popis modelu a možnost automatické kontroly popisu modelu.

Nevýhodami jsou náklady na vytvoření překladače, údržbu, výuku\ldots{}

Celkově nejsou příliš používány.

\section{Modelování náhodných procesů}
\label{sec:orgdef9d6c}
\subsection{Náhodná proměnná}
\label{sec:orgbf929cb}
Náhodná proměnná je taková veličina, která jako výsledek pokusů může nabýt
nějakou hodnotu, přičemž předem nevíme jakou.

Rozlišujeme diskrétní a spojité

Náhodné veličiny můžeme zadat distribuční funkcí nebo rozdělením
pravděpodobnosti.

\subsection{Diskrétní rozdělení pravděpodobnosti}
\label{sec:orgcc202f7}
Určuje vztah mezi možnými hodnotami náhodné veličiny \(x_i\) a jim příslušejícími
pravděpodobnostmi \(p_i = P(X = x_i)\).

Obecně platí vztah:
\begin{equation}
    \sum_{i=1}^\infty p_i = 1
\end{equation}

\section{Diskrétní simulace}
\label{sec:orgf4f0dc8}
\subsection{Úvod}
\label{sec:orgbecf5a4}
\subsection{Petriho sítě}
\label{sec:orge545f6e}
\subsection{Systémy hromadné obsluhy}
\label{sec:orgc31d431}
\subsection{SIMLIB}
\label{sec:orgb0fc8d5}

\section{Spojitá simulace}
\label{sec:org22d01e9}
\subsection{Úvod}
\label{sec:orgbc7abbf}
\subsection{Numerické metody}
\label{sec:org037f883}
\subsection{SIMLIB}
\label{sec:org6897213}

\section{Kombinovaná simulace}
\label{sec:orga689c73}
\subsection{Úvod}
\label{sec:org6d4c4aa}
\subsection{SIMLIB}
\label{sec:org2ddd9d8}
\end{document}
